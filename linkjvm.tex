%%%%%%%%%%%%%%%%%%%%%%%%%%%%%%%%%%%%%%%%%%%%%%%%%%%%%%%%
%%%
%%%  LaTeX template for Junior Journal article
%%%  Please don't change next few lines
%%%
%%%%%%%%%%%%%%%%%%%%%%%%%%%%%%%%%%%%%%%%%%%%%%%%%%%%%%%%

\documentclass{juniorjournal}
%\setcounter{page}{}
%\journaltitle{}
%\volume{}
%\pubyear{}
%\licensee{}
%\authorlist{}
%\doi{}
%\articlenumber{}
%\articletype{}

%%%%%%%%%%%%%%%%%%%%%%%%%%%%%%%%%%%%%%%%%%%%%%%%%%%%%%%%
%%% 
%%%  List custom packages below, if needed. 
%%%  Note that amsfonts,amstext,amssymb,amsmath,amsthm,amscd
%%%  bm,paralist,color,graphicx,array and subfigure packages
%%%  are all part of juniorjournal.cls
%%%
%%%%%%%%%%%%%%%%%%%%%%%%%%%%%%%%%%%%%%%%%%%%%%%%%%%%%%%%

\usepackage{listings}

\definecolor{javared}{rgb}{0.6,0,0} % for strings
\definecolor{javagreen}{rgb}{0.25,0.5,0.35} % comments
\definecolor{javapurple}{rgb}{0.5,0,0.35} % keywords
\definecolor{javadocblue}{rgb}{0.25,0.35,0.75} % javadoc

\lstset{language=Java,
basicstyle=\ttfamily,
keywordstyle=\color{javapurple}\bfseries,
stringstyle=\color{javared},
commentstyle=\color{javagreen},
morecomment=[s][\color{javadocblue}]{/**}{*/},
numbers=left,
numberstyle=\tiny\color{black},
stepnumber=2,
numbersep=10pt,
tabsize=4,
showspaces=false,
showstringspaces=false}

%%%%%%%%%%%%%%%%%%%%%%%%%%%%%%%%%%%%%%%%%%%%%%%%%%%%%%%%
%%%
%%%  Journal article basic info.
%%%  List all authors first and then their affiliations. 
%%%  Link author and affiliations with numbers in optional arguments.
%%%  For corresponding author use the form given below.
%%%  
%%%  For example:
%%%  \author[1]{Name1 Surname1}
%%%  \author[2, \corresp]{Name2 Surname2}\correspemail{email@edu.com}
%%%  \affil[1]{Institute1, Country1}
%%%  \affil[2]{Institute2, Country2}
%%%  
%%%%%%%%%%%%%%%%%%%%%%%%%%%%%%%%%%%%%%%%%%%%%%%%%%%%%%%%

\articletitle{LinkJVM - Java on the KIPR Link} % To break lines in long titles use \\

%\author[ ,\corresp]{}\correspemail{}
\author[1, \corresp]{Markus Klein}\correspemail{m@mklein.co.at}
\author[1]{Christoph Hackenberger}
\author[1]{Melanie Goebel}
\author[1]{Klaus Ableitinger}
\affil[1]{Vienna Institute of Technology}
%%%%%%%%%%%%%%%%%%%%%%%%%%%%%%%%%%%%%%%%%%%%%%%%%%%%%%%%
%%%
%%%  Main document.
%%%
%%%  To span floating objects (wide images, tables, etc.)
%%%  over twocolumn layout, use figure* 
%%%  or table* environments.
%%%
%%%  Examples:
%%%  \begin{figure*}	 \begin{table*}		
%%%   ...				  ...				
%%%  \end{figure*}		 \end{table*}		
%%%
%%%%%%%%%%%%%%%%%%%%%%%%%%%%%%%%%%%%%%%%%%%%%%%%%%%%%%%%

\begin{document}
\maketitle

\articleabstract{}
\keywords{Java, JVM, Botball, Library, Framework, Wrapper, KIPR Link, Libkovan, Robotic, JNI, SWIG, JamVM, GNU Classpath, Open Source}

\section{Introduction}
The purpose of Botball is to motivate students for building and programming autonomous robots.
Most new students to Botball do not have any experience with programming.
Therefore the way of developing the robot software has to be very beginner friendly and easy to use, but at the same time very powerful.

So what is the best language for Botball?
I do not think there is a best choice.
Every language has its own advantages and disadvantages.
C for example is a very powerful fast and processor-near language so it is quite good for a robot controller doing a lot of basic things such as controlling servos and motors, reading sensor values and so on. 
But for students who have not yet learned anything about programming or how a computer works, it would be very hard to learn and understand C code. 
If you want to develop a complex C program, you have to know how pointer arithmetic and memory access works.

Another well-known possibility is Java. 
Java is much easier to learn and also pretty powerful and reliable.
Of course it is not as fast as compiled languages such as C or C++, but in the most cases it will make no real difference.
Java is better for beginners, compared to C it does not require special knowledge about how a computer works.
Java is a higher level language.
It provides automatic garbage collection to handle memory management and offers an excellent set of basic data structures such as lists, queues and maps.
In C everything has to be built from scratch.
The object-oriented paradigm is also very helpful to structure the program to improve the maintainability.

In most schools in Austria students, like me, get started with programming using Java and not C or C++.
Therefore I would consider myself a good Java programmer, but I never wrote a bigger project using C.

Further alternatives are event-based frameworks using a scripting language such as node or in general scripting languages, but these can be implemented very easily, because the JVM(Java Virtual Machine) supports a huge amount of additional languages including javascript, scala and clojure.
%\lstinputlisting[language=Java]{listings/Debugger.java}

\section{State of the Art}
The JamVM which is currently preinstalled on the KIPR Link is not working.
I do not know why this version does not work yet.
If I try to run a program, the VM outputs \frqq segmentation fault \flqq and stops.
Unfortunately there is also no javac java compiler, which is working out-of-the-box.
So an alternative Java Runtime Environment (JRE) had to be installed to get the java compiler running.

\section{Design Approach}

\section{Java Environment}
LinkJVM includes a Java Runtime Environment(JRE) which consists of a lightweight Java Virtual Machine(JVM) and GNU Classpath an open source implementation of the java core classes.
Moreover it also offeres a Java compiler(javac) for compiling java source code as well as jar for packaging compiled class files.
\subsection{JamVM}

\begin{thebibliography}{500} % Include .bib (.bbl) references here

\end{thebibliography}

\end{document}
